\chapter{网络与外部设备}

\section{tcpdump和socket管理}
Linux内核提供了抓包(caputure packets)等服务作为内核网络套件的一部分。本节的目标包括对tcpdump进行定制化修改,以及对内核Socket的管理进行改进,从而实现更高的安全性和基于线程级的公平性(per-thread socket fairness)。

\subsection{功能目标}
\begin{enumerate}
    \item \textbf{定制化tcpdump抓包:}在tcpdump基础上进行修改,允许用户在不依赖tcpdump现有参数的情况下,自行传入过滤规则或者其他控制参数,以满足特定的抓包需求。
    \item \textbf{Socket公平性管理:}在内核中为每个线程提供独立的Socket资源上限或限流控制(fairness),确保多线程在执行网络操作时不会发生资源独占或过度使用的现象。
\end{enumerate}

\subsubsection{标准函数声明示例}
以下仅给出示例,用于说明如何编写对应接口的函数声明,具体细节可根据实际需求进行扩展。

\begin{lstlisting}[language=C, caption={示例:自定义tcpdump抓包接口}]
/**
 * @brief 使用自定义过滤规则对网络数据进行抓包
 * @param iface          需要进行抓包的网络接口名称
 * @param custom_filter  用户自定义的过滤表达式
 * @param buffer         存储抓取到的数据缓冲区
 * @param buffer_size    存储抓包数据的缓冲区大小
 * @return 成功时返回0,失败返回非0错误码
 */
int custom_tcpdump_capture(const char* iface, 
                           const char* custom_filter, 
                           void* buffer, 
                           size_t buffer_size);
\end{lstlisting}

\begin{lstlisting}[language=C, caption={示例:线程级别Socket公平性管理接口}]
/**
 * @brief 为特定线程配置Socket级别的公平管理策略
 * @param thread_id          需要配置的线程ID
 * @param max_socket_allowed 该线程允许打开的最大Socket数
 * @param priority_level     该线程的优先级(影响Socket分配策略)
 * @return 成功时返回0,失败返回非0错误码
 */
int configure_socket_fairness(pthread_t thread_id, 
                              int max_socket_allowed, 
                              int priority_level);
\end{lstlisting}

\subsection{tcpdump定制化抓包}
tcpdump 库基于 linux 内核的 libpcap 库实现,提供了对网络数据包的捕获和分析功能。我们可以利用 `<pcap.h>` 中的函数来实现这一功能。
具体而言,我们可以通过以下几步来完成定制化抓包:
\begin{enumerate}
    \item 使用 `pcap_open_live` 函数打开对应的网络设备。
    \item 使用 `pcap_compile` 函数编译用户自定义的过滤规则。
    \item 使用 `pcap_setfilter` 应用过滤规则。
    \item 使用 `pcap_loop` 函数开始进行抓包,并且进行进一步地处理。

\subsection{测试}
\begin{itemize}
    \item \textbf{定制抓包正确性:}验证是否确实只抓取了满足自定义过滤规则的数据。  
    \item \textbf{多Socket公平性:}测试多线程同时创建并使用Socket时,是否能按照规划的最大数目和优先级进行分配,并观察线程间资源竞争的情况。
    \item \textbf{性能指标:}在系统负载较高时统计p99延迟(多Socket读写操作)。
\end{itemize}

\section{高速通信库:NCCL}
由于单节点计算能力的提升,分布式计算对通信带宽和延迟提出了极高要求。传统以太网通信常常无法满足此种高吞吐需求。NCCL (NVIDIA Collective Communications Library) 提供了GPU间高速通信接口,可广泛应用于深度学习训练场景。下文给出了相关参考文档以及源代码仓库:
\begin{itemize}
    \item \href{https://docs.nvidia.com/deeplearning/nccl/user-guide/docs/examples.html#communicator-creation-and-destruction-examples}{Nvidia NCCL examples}
    \item \href{https://github.com/NVIDIA/nccl}{Nvidia NCCL GitHub repo}
\end{itemize}

\subsection{功能目标}
\begin{enumerate}
    \item 在多GPU环境下,使用NCCL实现简单的信息传输(如广播、All-Reduce等),验证数据传输的正确性。
    \item 与以太网通信方式进行带宽和延迟对比,观察速度提升和正确性差异。
\end{enumerate}

\subsubsection{标准函数声明示例}
\begin{lstlisting}[language=C, caption={示例:NCCL通信例程接口}]
/**
 * @brief 使用NCCL执行多GPU数据广播操作
 * @param data         需要广播的数据缓冲区指针
 * @param count        数据元素的个数
 * @param root         广播发送方的GPU ID
 * @param comm         NCCL通信器
 * @return ncclSuccess表示成功,其他错误码请参照NCCL官方文档
 */
ncclResult_t nccl_broadcast_data(void* data, 
                                 size_t count, 
                                 int root, 
                                 ncclComm_t comm);
\end{lstlisting}

\subsection{测试项与测试方式}
\begin{itemize}
    \item \textbf{正确性测试:}不同GPU间进行广播、聚合(All-Reduce)后,结果是否一致,误差是否在可接受范围内(通常为浮点运算误差)。
    \item \textbf{性能测量:}分别采用NCCL和以太网(TCP/IP)进行跨节点通信,记录带宽、吞吐量、延迟等指标,对比二者差异。
\end{itemize}

\subsection{更多细节}
具体而言, nccl 库支持的是 \textbf{GPU} 之间的通信。所以具体而言,在当前的接口下,我们总共需要这样几步来完成工作:
\begin{enumerate}
    \item 为当前的 GPU 创建对应的 cuda stream.
    \item 在 gpu 上分配对应的内存空间。
    \item 把 主存中的数据拷贝到 GPU 上。
    \item 调用 nccl 库进行广播。
    \item cuda 间进行同步,并且释放 gpu 上空间。
\end{enumerate}
在当前的接口下,这样应该足够完成工作了。

\section{数据平面开发套件:DPDK}
DPDK( Data Plane Development Kit )是一套在用户态加速网络收发的开发工具包。它将网络协议栈从内核态卸载到用户态中运行,从而极大减少特权级切换(toggle between privileged levels)并提升网络吞吐量。本节在有限的复杂度下,演示如何利用DPDK实现多进程带宽平衡的QoS管理。

\subsection{功能目标}
\begin{enumerate}
    \item 在用户态基于DPDK实现最简化的网络收发功能。
    \item 设计一个QoS管理策略,使得多个进程共享网络带宽时,能根据设定的策略进行带宽分配并保证平衡。
\end{enumerate}

\subsubsection{标准函数声明示例}
\begin{lstlisting}[language=C, caption={示例:DPDK QoS管理接口}]
/**
 * @brief 在DPDK环境下为队列配置带宽限速
 * @param queue_id       DPDK端口队列编号
 * @param rate_limit     限速值(单位:Mbps)
 * @return 0表示成功,非0表示发生错误
 */
int dpdk_qos_configure_rate_limit(uint16_t queue_id, 
                                  uint32_t rate_limit);

/**
 * @brief 发送数据包入口函数
 * @param buffer         需要发送的包数据指针
 * @param len            包长度
 * @param queue_id       发送使用的DPDK端口队列编号
 * @return 发送成功的数据包数,出现错误返回负值
 */
int dpdk_send_packet(const uint8_t* buffer, 
                     size_t len, 
                     uint16_t queue_id);
\end{lstlisting}

\subsection{测试项与测试方式}
\subsubsection{测试项}
\begin{itemize}
    \item \textbf{基本发送接收功能:}在用户态能够正确收包、发包,确保功能可用。
    \item \textbf{多进程带宽平衡:}当多个进程请求同时发送数据时,观察各进程的实际带宽分配是否符合设定的QoS管理策略。
    \item \textbf{性能指标:}记录最大吞吐量、p99延迟,比较在无QoS管理与有QoS管理时的性能差异。
\end{itemize}

\subsubsection{测试方法}
\begin{enumerate}
    \item \textbf{功能正确性:}  设置单进程收发场景,发送固定大小的数据包,检查收发统计信息。
    \item \textbf{多进程带宽平衡:}  并行启动多个进程,各自不断发送数据包。使用dpdk\_qos\_configure\_rate\_limit函数为不同队列设置不同的速率限制(不同版本可能略有不同,可以使用自己期望的版本并且告知助教后进行调整),测试带宽使用情况是否符合预期配额。
    \item \textbf{压力测试:} 持续提高数据发送速率或缩短发包间隔,检测稳定性、丢包率和延迟情况,测试不同QoS设置的效果。
\end{enumerate}
