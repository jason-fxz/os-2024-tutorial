\chapter{项目环境准备}

\section{基准代码}
本系列的基准代码规则如下:
\begin{enumerate}
    \item \textbf{Linux}: 任何体系结构下高于5.10.20版本的Linux内核都是可行的选项,你可以从\url{https://www.kernel.org/}中选择合适的版本。对于\texttt{initramfs},你可以选择任何一个发行版。如果你选择直接使用qcow或者其他完全预制好的Ubuntu镜像,你需要手动在内部更换内核以确保你的代码能够生效。
    \item \textbf{QEMU}: 版本>=7。
    \item \textbf{EDK2}: 跟随\url{https://github.com/tianocore/edk2}选择合适的版本。其中Ubuntu\_GCC5并不意味着GCC版本必须是5。
\end{enumerate}

\section{建议的代码管理方式}

建议单独开一个 git 本地仓库用于存放 Linux 内核,init commit 提交你下载的 Linux 内核代码。之后需要修改内核代码时,\textbf{开不同的分支} 便于代码管理。最后提交作业时,可使用 git diff 命令生成 patch 文件
\begin{lstlisting}[language=bash]
# 生成当前分支与主分支差异的补丁文件
git diff main > changes.patch

# 生成两个提交之间的差异
git diff <commit1> <commit2> > commit_diff.patch

# 生成两个分支之间的差异
git diff <branch1> <branch2> > branch_diff.patch
\end{lstlisting}

然后将生成的 patch 文件提交到作业中。

\section{在QEMU中启动Linux镜像}
\begin{remark}
    \label{qcow-ubuntu-image}
简单起见你可以直接使用预制作好的\texttt{qcow} Ubuntu镜像作为磁盘镜像由qemu直接引导,具体教程:\url{https://documentation.ubuntu.com/public-images/en/latest/public-images-how-to/launch-qcow-with-qemu/}。如果你选用这个方案,你只需要确保QEMU正常安装即可。
\end{remark}

\begin{remark}
    本区域中.x.x部分请填入自己对应的版本号。
\end{remark}



\subsection{QEMU安装(二选一)}
\subsubsection{软件包安装}
\begin{lstlisting}[language=bash]
sudo apt-get install qemu-system-x86
qemu-system-x86_64 --version
\end{lstlisting}

\subsubsection{源码编译安装}
\begin{lstlisting}[language=bash]
# 安装依赖
sudo apt-get install git libglib2.0-dev libfdt-dev libpixman-1-dev zlib1g-dev ninja-build

# 下载编译
mkdir ~/kvm && cd ~/kvm
wget https://download.qemu.org/qemu-7.2.0.tar.xz
tar -xf qemu-7.2.0.tar.xz
cd qemu-7.2.0
./configure --target-list=x86_64-softmmu
make -j$(nproc)
\end{lstlisting}

\subsubsection{常见依赖问题}
\begin{itemize}
    \item \textbf{ERROR: Cannot find Ninja} → 执行 \texttt{sudo apt install ninja-build}
    \item \textbf{ERROR: glib-2.56 required} → 执行 \texttt{sudo apt install libglib2.0-dev}
    \item \textbf{Dependency "pixman-1" not found} → 执行 \texttt{sudo apt install libpixman-1-dev}
\end{itemize}

\subsection{Linux内核编译}
\begin{lstlisting}[language=bash]
# 安装依赖
sudo apt-get install git fakeroot build-essential ncurses-dev xz-utils libssl-dev bc flex libelf-dev bison
mkdir ~/kernel && cd ~/kernel
wget https://git.kernel.org/pub/scm/linux/kernel/git/stable/linux.git/snapshot/linux-5.x.x.tar.gz
tar -xf linux-5.x.x.tar.gz
cd linux-5.x.x

make defconfig    # 使用默认配置
make -j$(nproc)   # 开始编译
\end{lstlisting}

\textbf{编译输出文件}: \texttt{arch/x86/boot/bzImage}

在编译过程中,可能会发生编译错误的情况。这可能是 GCC 版本过高导致的。这里给出一种可行的组合: Linux-5.19.17, 使用 GCC-12 进行编译,操作如下:

\begin{lstlisting}[language=bash]
sudo apt install gcc-12 g++-12 # 安装gcc-12
sudo update-alternatives --install /usr/bin/gcc gcc /usr/bin/gcc-12 70
sudo update-alternatives --install /usr/bin/g++ g++ /usr/bin/g++-12 70 # 切换 gcc 默认版本
\end{lstlisting}
经过这些操作后,应该可以正常编译 Linux 内核了。

\subsection{BusyBox}
\subsubsection{直接安装}
直接使用你的开发环境自带的包管理器安装busybox,然后在后面制作initramfs的过程中添加这两步:

先把busybox拷贝进initramfs:
\begin{lstlisting}[language=bash]
cp $(which busybox) "${INITRAMFS_DIR}/bin/busybox"
\end{lstlisting}

然后在系统启动脚本里添加:
\begin{lstlisting}[language=bash]
/bin/busybox --install -s /bin
\end{lstlisting}
\subsubsection{手动编译安装}
这里可以采用其他发行版(例如Ubuntu发行版等)。
\begin{lstlisting}[language=bash]
mkdir ~/kvm && cd ~/kvm
wget https://busybox.net/downloads/busybox-1.x.x.tar.bz2
tar -xf busybox-1.x.x.tar.bz2
cd busybox-1.x.x

# 配置命令
make menuconfig
# 选中: Settings > Build Options > [*] Build static binary
# 取消: Shells > [ ] Job control

make -j$(nproc) && make install
\end{lstlisting}
如果在编译过程中出现错误( TCA\_CBQ\_MAX undeclared ),一种解决方式是直接删除出错的 “tc.c” 文件。之后就可以正常编译了(版本 1.32.0 可以用此方法解决)。

\subsection{安装 edk2}
若不需要给 \texttt{edk2} 添加自定义功能,你可以直接使用包管理器安装预编译的 \texttt{edk2} 。\texttt{Ubuntu 24.04}下可以执行:
\begin{lstlisting}[language=bash]
sudo apt-get install ovmf
\end{lstlisting}

安装完成后,你可以在 \texttt{/usr/share/OVMF} 目录下找到 \texttt{OVMF\_CODE\_4M.fd} 和 \texttt{OVMF\_VARS\_4M.fd} 两个文件,将它们的路径填入后文 \texttt{qemu-system-x86\_64} 的启动参数中即可使用。

\subsection{EDK2编译(简述)}

EDK2的可编译源码在git submodule里,因此相对简单可靠的方式就是直接按照官网所说的那样:
\begin{lstlisting}[language=bash]
git clone -b stable/202408 https://github.com/tianocore/edk2.git
cd edk2
git submodule update --init
\end{lstlisting}

在编译之前,需要在系统里安装一些基础的包(例如base-devel uuid、nasm、acpica),然后就可以编译BaseTools。

\begin{lstlisting}[language=bash]
    sudo apt install iasl nasm
\end{lstlisting}

在笔者实践的过程中,遇到了默认安装的 nasm 版本过低,导致后续编译出错的问题。读者可参考后文 \ref{compile-edk2} 的内容,自行编译 nasm。

\begin{lstlisting}[language=bash]
source edksetup.sh
make -C BaseTools -j$(nproc)
\end{lstlisting}
然后编辑\texttt{Conf/target.txt}(参考\url{https://github.com/tianocore/tianocore.github.io/wiki/How-to-build-OVMF}):
\begin{lstlisting}
ACTIVE_PLATFORM       = OvmfPkg/OvmfPkgIa32X64.dsc
TARGET_ARCH           = IA32 X64
TOOL_CHAIN_TAG        = GCC5
\end{lstlisting}
最后,执行:
\begin{lstlisting}[language=bash]
build
\end{lstlisting}
然后就可以在\texttt{Build/Ovmf3264/DEBUG\_GCC5/FV/}目录找到\texttt{OVMF\_CODE.fd}和\texttt{OVMF\_VARS.fd}两个编译出的OVMF文件。(这里的Ovmf3264和DEBUG\_GCC5和编译选项有关,也是在\texttt{Conf/target.txt}里面配置。)

\subsection{制作initramfs}
\subsubsection{创建启动脚本}
\begin{lstlisting}[language=bash, language=bash, showstringspaces=false]
cd busybox-1.x.x/_install/
mkdir -p proc sys dev tmp
cat > init << 'EOF'
#!/bin/sh
mount -t proc none /proc
mount -t sysfs none /sys
mount -t tmpfs none /tmp
mount -t devtmpfs none /dev
echo "Hello Linux!"
sh
poweroff -f
EOF
chmod +x init
\end{lstlisting}

\subsubsection{按需添加测试程序}

initramfs 就是我们简易版 Linux 的文件系统,我们可以在里面添加测试程序。如果是用 C 语言编写,在编译时添加 \texttt{-static} 选项静态编译,然后添加到 \texttt{busybox-1.x.x/\_install/bin} 里即可。如果程序难以静态编译,则可以使用\hyperref[appendix:dynamic-library]{附录}中的脚本将程序所依赖的动态链接库添加到 initramfs 里。


\subsubsection{打包文件系统}
\begin{lstlisting}[language=bash]
cd busybox-1.x.x/_install/
find . -print0 | cpio --null -ov --format=newc | gzip -9 > ../initramfs.cpio.gz
\end{lstlisting}

\subsection{启动QEMU}
\subsubsection{图形界面启动}
\begin{lstlisting}[language=bash]
qemu-system-x86_64 \
    -kernel ./kernel/linux-5.x.x/arch/x86/boot/bzImage \
    -initrd ./kvm/busybox-1.x.x/initramfs.cpio.gz \
    -append "init=/init"
\end{lstlisting}

\subsubsection{字符界面启动}
\begin{lstlisting}[language=bash, showstringspaces=false]
qemu-system-x86_64 \
    -kernel ./kernel/linux-5.x.x/arch/x86/boot/bzImage \
    -initrd ./kvm/busybox-1.x.x/initramfs.cpio.gz \
    -nographic \
    -append "init=/init console=ttyS0"
\end{lstlisting}

使用EDK2的UEFI模拟器启动时,需要使用\ref{run-edk2}中的命令。

\subsection{验证}
\begin{itemize}
    \item 检查内核版本: \texttt{uname -a}
    \item 查看启动参数: \texttt{cat /proc/cmdline}
    \item 测试基本命令: \texttt{ls}, \texttt{mount}
\end{itemize}

\section{Linux环境部署 EDK2 开发环境}

\subsection{编译 EDK2}

python的版本要求>=3.11,

\subsubsection{下载源码}
\begin{lstlisting}[language=bash]
git clone https://github.com/tianocore/edk2.git
cd edk2
git submodule update --init  #大概需要几分钟
cd ..
\end{lstlisting}

\subsubsection{网上找到的可能需要安装的一堆东西} \label{compile-edk2}

\begin{lstlisting}[language=bash]
sudo apt-get install build-essential uuid-dev
sudo apt-get install uuid-dev nasm bison flex
sudo apt-get install libx11-dev x11proto-xext-dev libxext-dev
\end{lstlisting}

将 NASM 更新到 2.15.x 以上版本,当前的edk2项目需要依赖nasm2.15以上版本,否则编译会报错
nasm各版本下载链接:http://www.nasm.us/pub/nasm/releasebuilds/?C=M;O=D
(特别注意,推荐下载 .tar.xz 格式的 source code, .zip 解压后的文件似乎都是 CRLF 格式,可能会导致编译错误)

\begin{lstlisting}[language=bash]
cd nasm-2.16
./autogen.sh
./configure
make
make install
nasm --version
\end{lstlisting}

\subsubsection{make一下BaseTools}

然后cd到edk2/BaseTools下
\begin{lstlisting}[language=bash]
make
\end{lstlisting}

显示ok则成功。
注意这里所有的文件应该是LF,不能是CRLF,不然会出问题。可以先检查一下。

然后到edk2的conf目录下,创建target.txt(template在readme里面)
然后修改:
TARGET\_ARCH           = X64
TOOL\_CHAIN\_TAG        = GCC5

\subsubsection{编译UEFI模拟器和UEFI工程}
先到edk2目录下,然后执行
\begin{lstlisting}[language=bash]
source edksetup.sh
build
\end{lstlisting}

然后到target.txt里面改一下:
ACTIVE\_PLATFORM       = OvmfPkg/OvmfPkgX64.dsc

\subsection{运行}

\label{run-edk2}

\subsubsection{只运行UEFI环境}
在一开始的任务中,完全没有必要拉起一个完整的系统环境,甚至连Linux系统内核镜像都不需要启动,只需要全程在UEFI环境里执行。

此时的UEFI环境,简单来讲由两部分构成:
\begin{enumerate}
	\item OVMF文件:用于在虚拟机中,提供UEFI固件,从而启动UEFI环境。
	\item 若干.efi文件:UEFI环境下的可执行文件
\end{enumerate}

之前所说的那个`build`指令,用处就是编译OVMF固件。你可能还希望使用这个命令编译一个更强大的UEFI Shell
\begin{lstlisting}[language=bash]
build -p ShellPkg/ShellPkg.dsc
\end{lstlisting}

然后你会找到一个\texttt{AcpiViewApp.efi}文件(可以使用\texttt{find . -name "AcpiViewApp.efi"}来寻找它),把它挂载进文件系统,你就可以在UEFI Shell里使用它来执行EDK2预先写好的一个AcpiView。详细的示例,参考\hyperref[appendix:launchuefi]{附录}

进入 \texttt{UEFI Shell} 后,执行下列命令:

\begin{lstlisting}[language=bash]
    fs0:            # go to `uefi` directory
    AcpiViewApp.efi # run AcpiView
    reset           # if you add `-no-reboot` to the qemu command, this will exit the UEFI Shell
\end{lstlisting}


\begin{remark}
    在 \texttt{UEFI} 环境下,如果 \texttt{backspace} 键无法使用,可以使用 \texttt{Ctrl+H} 键来代替。
\end{remark}

\begin{remark}
    将 UEFI 指令写入 \texttt{uefi} 目录下的 \texttt{startup.nsh} 文件中,可以实现开机自动运行。
\end{remark}


\subsubsection{运行 UEFI 环境和自定义 Linux}

首先将编译好的内核镜像和 \texttt{initramfs} 拷贝到 \texttt{uefi} 目录下(也可以软链接)。然后执行:

\begin{lstlisting}[language=bash]
    cp /path/to/kernel/arch/x86/boot/bzImage uefi/bzImage
    cp /path/to/initramfs.cpio.gz uefi/initramfs.cpio.gz
    qemu-system-x86_64 \
      -drive if=pflash,format=raw,readonly=on,file={Your OVMF_CODE.fd here} \
      -drive if=pflash,format=raw,file={Your OVMF_VARS.fd here} \
      -drive file=fat:rw:./uefi,format=raw,if=ide,index=0 \
      -nographic
\end{lstlisting}

进入 \texttt{UEFI} 环境之后,执行下列命令:

\begin{lstlisting}[language=bash]
    fs0:            # go to `uefi` directory
    # boot the kernel
    bzImage initrd=initramfs.cpio.gz init=/init console=ttyS0
    reset                   # in case the kernel does not boot
\end{lstlisting}

启动 \texttt{Linux} 系统后,可以通过如下命令将 \texttt{uefi} 文件夹挂载到 \texttt{/mnt/efi} 目录:

\begin{lstlisting}[language=bash]
    mount /dev/sda1 /mnt/efi
\end{lstlisting}

若使用该方法运行虚拟机,在 \texttt{Linux} 系统使用 \texttt{reboot -f} 重启虚拟机,会重新进入 \texttt{UEFI Shell} 环境。

\subsubsection{运行 UEFI 环境和 Ubuntu}

首先根据前文\ref{qcow-ubuntu-image} 制作一个 Ubuntu 镜像 \texttt{ubuntu.img}。

然后运行 \texttt{qemu} :

\begin{lstlisting}[language=bash]
    qemu-system-x86_64 \
    -machine q35 \
    -m 8G \
    -smp 4 \
    -snapshot \
    -drive if=pflash,format=raw,unit=0,file={Your OVMF_CODE.fd here},readonly=on \
    -drive if=pflash,format=raw,unit=1,file={Your OVMF_VARS.fd here} \
    -drive file="/path/to/ubuntu.img",format=qcow2,if=virtio \
    -drive file=fat:ro:"/path/to/folder/uefi",format=raw,if=ide,index=1 \
    -nographic \
    -net none \
    -serial mon:stdio
\end{lstlisting}

注意到我们将一个文件夹 \texttt{uefi} 也挂载进 \texttt{UEFI} 环境中。它在 \texttt{UEFI Shell} 中的路径为 \texttt{fs0:}。我们可以在把自己编写的 \texttt{UEFI} 程序放在这里。

而 \texttt{Ubuntu} 启动文件夹在 \texttt{UEFI} 中的路径为 \texttt{fs1:}。进入这个目录后,有两种方法可以启动 \texttt{Ubuntu}:

\begin{lstlisting}[language=bash]
    /EFI/BOOT/BOOTX64.EFI
    # or
    /EFI/ubuntu/grubx64.efi
\end{lstlisting}

进入 \texttt{Ubuntu} 系统后,\texttt{Ubuntu} 的启动文件夹位于 \texttt{/boot/efi} 下。

\begin{remark}
    上述 qemu 启动命令中,-snapshot 选项表示虚拟机关机后,对虚拟机状态的修改不会保存到磁盘上。该选项非必须。
\end{remark}